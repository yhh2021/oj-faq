\documentclass[twocolumn, a5paper]{article}
\usepackage[scale=.8]{geometry}
\usepackage[]{xeCJK}

\title{OJ调试指南}
\author{颜汇杭}

\begin{document}
\maketitle	
\thispagestyle{empty}

\section{编译错误}

编译错误的意思是,代码有语法错误,完全运行不了。

请在Jupyter Notebook中本地调试。

\section{Runtime Error}

RE的意思是,代码语法没问题,但是运行时出现异常,运行到一半就炸了。

请在Jupyter Notebook中本地调试,并使用题目中的示例输入。

\section{Wrong Answer}

WA的意思是,代码语法没问题,运行时也没出现异常,但是结果不对。

请在Jupyter
Notebook中本地调试,并使用题目中的示例输入,与示例输出作比较。

\section{一些常见问题}

\begin{enumerate}
	\item 题目要求使用小数,却使用了int,导致RE;
	\item 在input()和print()等地方输出了题目没有要求的,通常是给人类读的,一些提示性文字,导致WA;
	\item
			题目输入的数之间用空格分割,但是却用多个input()读入,而不是input().split()。
\end{enumerate}

\end{document}
