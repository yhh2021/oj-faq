\documentclass[twocolumn, a5paper]{article}
\usepackage[scale=.81]{geometry}
\usepackage[]{xeCJK}

\title{OJ调试指南}
\author{颜汇杭}

\begin{document}
\maketitle	
\thispagestyle{empty}

\section{编译错误}

编译错误的意思是,代码有语法错误,完全运行不了。

请在Jupyter Notebook中本地调试。

\section{Runtime Error}

RE的意思是,代码语法没问题,但是运行时出现异常,运行到一半就炸了。

请在Jupyter Notebook中本地调试,并使用题目中的示例输入。

\section{Wrong Answer}

WA的意思是,代码语法没问题,运行时也没出现异常,但是结果不对。

请在Jupyter
Notebook中本地调试,并使用题目中的示例输入,与示例输出作比较。

\section{常见问题}

\begin{enumerate}
	\item 题目要求使用小数,却使用了int,导致RE。
	\item
			在input()和print()等地方输出了题目没有要求的一些给人类读的提示性文字,导致WA。
	\item
			题目输入的数之间用空格分割,但是却用多个input()读入,而不是input().split()。
	\item
			只在平台上敲代码,不在本地测试。这样是很难得到正确答案的。程序需要调试,不是一下子写好的。
\end{enumerate}

\section{资料}

对一些东西不熟悉时,可以参考这些资料:

\begin{enumerate}
		\item 在Jupyter
				Notebook中使用help()。如help(range)将得到关于range函数的简要说明。
				\item
						官方文档:https://docs.python.org/3。点右上角的搜索框可以搜索。比如,想查格式化字符串的话,可以搜索“format”,在下面找到“7.
						输入和输出”,点进去查看。
				\item
						官方文档的“Python入门”一节(/tutorial)是非常好的入门教材,可以参考。
\end{enumerate}

\end{document}
